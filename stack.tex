\documentclass[10pt,letterpaper]{article}
\textwidth = 6.5in
\textheight = 9in
\hoffset=-.75in
\voffset=-.8in


\usepackage{alltt}
\usepackage{tcolorbox}
%\tcbset{colback=green!20!white}


\usepackage[pass]{geometry}
\usepackage{color}   % Necessary for colored links

\usepackage[hypcap=true]{caption}       % point to top of figure rather than caption in \hyperref

%\usepackage[hypertex]{hyperref}
\usepackage{hyperref}
\hypersetup{
    colorlinks=true,    %set true if you want colored links
    linkcolor=blue      %choose some color if you want links to stand out
}

\usepackage{lastpage}
\usepackage{fancyhdr}
\usepackage{sectsty}
\usepackage{amsmath}
\usepackage{scrextend}

\usepackage{listings}
\usepackage{xcolor}
\usepackage{graphicx}
\usepackage{epsfig}

%\graphicspath{ {./images/} }
\usepackage{tikz}
\usepackage{tikz-qtree}
\usepackage{tikz-timing}
	\usetikzlibrary{arrows.meta}
	%\usetikztiminglibrary[simple]{advnodes}
	\usetikztiminglibrary{advnodes}
	\usetikzlibrary{automata, positioning, arrows}
\usepackage{enumitem}
\usepackage{placeins}



\lstset{numbers=left, numberstyle=\small, stepnumber=1, numbersep=5pt, basicstyle=\footnotesize\ttfamily}


\newcommand{\theListingFontFamily}{\ttfamily\footnotesize}

\lstnewenvironment{tty}
    {\lstset{language=sh, %
        numbers=left,numberstyle=\tiny,stepnumber=1,numbersep=8pt, %
        breaklines=true, %
        frame=single, %
        showtabs=false, %
        basicstyle=\theListingFontFamily, %
        showstringspaces=false, %
        tabsize=4, %
        showlines=true, %  show all blank lines
        basicstyle=\theListingFontFamily}}
    {}


\newcommand{\reducedstrut}{\vrule width 0pt height .9\ht\strutbox depth .9\dp\strutbox\relax}
\newcommand\scb[2]{\begingroup\fboxsep=0pt\colorbox{#1}{\reducedstrut#2\/}\endgroup}




\sectionfont{\Large\sf\bfseries}
\subsectionfont{\large\sf\bfseries}

\pagestyle{fancy}
% supress normal headings and footters
\fancyhf{}
% remove the heading rule
\renewcommand{\headrulewidth}{0pt}
\renewcommand{\footrulewidth}{1pt}


\lfoot{{\sf\scriptsize Copyright \copyright\ 2020,2022 John Winans.  All Rights Reserved}\\
{\scriptsize\FooterText}}
%\lfoot{\scriptsize\FooterText}

\rfoot{Page \thepage\ of \pageref*{LastPage}}

% Sub-footer that shows the VCS Header in the lfoot defined above
\ifdefined\GitFileName
    \newcommand{\FooterText}{\tt \GitFileName\\
\GitDescription}
\else
    \newcommand{\FooterText}{\emph{--UNKNOWN--}}
\fi


\setlength{\parindent}{0pt}
\setlength{\parskip}{.51em}


%%%%%%%%%%%%%%%%%%%%%%%%%%%%%%%%%%%%%%%%%%%%%%%%%%%%%%%%%%%%%%%%%%%%
\definecolor{c_lightblue}{HTML}{B0E0FF}
\definecolor{c_lightred}{HTML}{FFE0E0}
\definecolor{c_lightyellow}{HTML}{FFF0a0}
\definecolor{c_lightgreen}{HTML}{60FF60}
\definecolor{c_lightgray}{HTML}{C0C0C0}
%%%%%%%%%%%%%%%%%%%%%%%%%%%%%%%%%%%%%%%%%%%%%%%%%%%%%%%%%%%%%%%%%%%%

\tcbset{colback=c_lightgreen!20!white}

%%%%%%%%%%%%%%%%%%%%%%%%%%%%%%%%%%%%%%%%%%%%%%%%%%%%%%%%%%%%%%%%%%%%
%%%%%%%%%%%%%%%%%%%%%%%%%%%%%%%%%%%%%%%%%%%%%%%%%%%%%%%%%%%%%%%%%%%%
%%%%%%%%%%%%%%%%%%%%%%%%%%%%%%%%%%%%%%%%%%%%%%%%%%%%%%%%%%%%%%%%%%%%
%%%%%%%%%%%%%%%%%%%%%%%%%%%%%%%%%%%%%%%%%%%%%%%%%%%%%%%%%%%%%%%%%%%%
%%%%%%%%%%%%%%%%%%%%%%%%%%%%%%%%%%%%%%%%%%%%%%%%%%%%%%%%%%%%%%%%%%%%

\title{Stack Organization}
\author{John Winans}
%\date{January 2022}
\date{}

\begin{document}

\maketitle

\thispagestyle{fancy}
%\thispagestyle{plain}

\section{Overview}

A \textbf{stack} is an area of memory that is used to store data using what 
is called a \textbf{LIFO} (Last In First Out) data structure.
Data is inserted into a stack by using a \textbf{push} operation and removed by 
using a \textbf{pop} operation.

\section{Ascending/Descending (Direction of Growth)}

During the initialization of an application, a section of memory is 
reserved for the program's \textbf{call stack} (or simply the {\em program's stack.}
The starting address of the stack is loaded into a \textbf{stack pointer}
register that is then used to keep track of the location of the ``top'' of the stack.

The section of memory reserved for a stack can grow (or ``be filled with data'') 
from the highest address to the lowest or from the lowest address to the highest.%
\footnote{Note that the direction of a stack's growth has nothing to do 
with endianness.}

When a stack is filled from the highest address to the lowest, the stack
is referred to as a \textbf{descending} stack. When data elements are pushed onto
a descending stack, the stack pointer is {\em decremented} by the number of 
bytes pushed. When data elements are popped out of a descending stack,
the stack pointer is {\em incremented} by the number of bytes popped.

When a stack is filled from the lowest address to the highest, the
stack is referred to as an \textbf{ascending} stack.  
When data elements are pushed into an ascending stack, the stack pointer is 
{\em incremented} by the number of bytes pushed and when data elements are 
popped out of an ascending stack, it is {\em decremented} by the number of bytes popped.


\section{Full vs Empty}

A stack can also be classified as being full \textbf{full} or \textbf{empty}.

In an empty stack, the stack pointer points to the next free (empty) 
location on the stack.  
When pushing data into an {\em empty} stack, the data is first stored in 
the stack at the address in the stack pointer and then the stack pointer 
is adjusted to reflect its new location.

In a Full stack, the stack pointer points to the topmost item in the stack
(the last item pushed into it.)
When pushing data into a {\em Full} stack, 
the stack pointer is first adjusted to reflect its new location and then
the data is stored in the stack at the address in the stack pointer.

In other words, this answers the question ``is the stack pointer
pointing at a location where the {\em last} value that has been pushed into 
the stack or does it point at a location where {\em next} pushed value 
would go?




\section{Conclusion}

This allows for four different types of stack organizations:

\begin{itemize}
\item Full Descending
\item Empty Descending
\item Full Ascending
\item Empty Ascending
\end{itemize}

Most microprocessors use a Full, Descending stack starting at the 
highest memory address.






%%%%%%%%%%%%%%%%%%%%%%%%%%%%%%%%%%%%%%%%%%%%%%%%%%%%%%%%%%%%%%%%%%%%%%%%%%%%
%%%%%%%%%%%%%%%%%%%%%%%%%%%%%%%%%%%%%%%%%%%%%%%%%%%%%%%%%%%%%%%%%%%%%%%%%%%%
%%%%%%%%%%%%%%%%%%%%%%%%%%%%%%%%%%%%%%%%%%%%%%%%%%%%%%%%%%%%%%%%%%%%%%%%%%%%
%%%%%%%%%%%%%%%%%%%%%%%%%%%%%%%%%%%%%%%%%%%%%%%%%%%%%%%%%%%%%%%%%%%%%%%%%%%%
%%%%%%%%%%%%%%%%%%%%%%%%%%%%%%%%%%%%%%%%%%%%%%%%%%%%%%%%%%%%%%%%%%%%%%%%%%%%
%%%%%%%%%%%%%%%%%%%%%%%%%%%%%%%%%%%%%%%%%%%%%%%%%%%%%%%%%%%%%%%%%%%%%%%%%%%%


\section{Examples}

This section presents an example of push and pop operations for each of the four
types of stack implementation.

The reader should observe that each type of {\em push} instruction varies only
in that the SP register is adjusted either before (full) or after (empty) data 
is stored into the stack memory and such adjustment is either done by adding 
(ascending) or subtracting (descending) the number of bytes representing the 
size of the value being pushed.

The reader should also observe that each type of {\em pop} instruction varies only
in that the SP register is adjusted either before (empty) or after (full) data
is read from the stack memory and such adjustment is performed by either adding 
(descending) or subtracting (ascending) the number of bytes representing the 
size of the value being popped.







\subsection{Descending Stacks}

This section provides a detailed discussion of push and pop operations
on descending stacks.


%%%%%%%%%%%%%%%%%%%%%%%%%%%%%%%%%%%%%%%%%%%%%%%%%%%%%%%%%%%%%%%%%%%%%%%%%%%%
%%%%%%%%%%%%%%%%%%%%%%%%%%%%%%%%%%%%%%%%%%%%%%%%%%%%%%%%%%%%%%%%%%%%%%%%%%%%
\subsubsection{Full Descending Stack: Push}

If a {\em Full Descending} stack is employed on a {\em big-endian} machine
with the following memory contents:
\begin{verbatim}
000000e0  c9 b9 6e 41 b2 85 af 8d  c8 71 dd 31 12 2b 2c f9
000000f0  78 31 b8 c2 a7 d0 de 27  47 24 5b b3 e0 e9 3b e2
00000100  91 de b5 9e e1 90 d4 b5  06 d7 66 ad 79 dc 17 fb
\end{verbatim}
and the SP (stack pointer) register is currently set to \verb@0x00000100@ then 
a {\em push} of the 32-bit value \verb@0x01021a04@ onto the stack would be 
performed like this:

\begin{tabular}{ll}
{\tt SP $\leftarrow$ SP - 4}              & // {\em pre}-decrement using operand size SP=\verb@0x000000fc@\\
{\tt mem32(SP) $\leftarrow$ 0x01021a04}     & // store the 32-bit value into memory
\end{tabular}

Resulting in changing the contents of the four bytes at address 
\verb@0x000000fc@ to this:
\begin{alltt}
000000e0  c9 b9 6e 41 b2 85 af 8d  c8 71 dd 31 12 2b 2c f9
000000f0  78 31 b8 c2 a7 d0 de 27  47 24 5b b3 \scb{c_lightyellow}{01 02 1a 04}
00000100  91 de b5 9e e1 90 d4 b5  06 d7 66 ad 79 dc 17 fb
\end{alltt}
\ldots\ and SP=\verb@0x000000fc@.


\begin{tcolorbox}
Note that if this same example were run on a {\em little-endian} machine, 
the only difference would be that the four bytes starting at memory address 
\verb@0x000000fc@ would have been set to \verb@04 1a 02 01@ (rather than 
\verb@01 02 1a 04@) like this:

\begin{alltt}
000000f0  78 31 b8 c2 a7 d0 de 27  47 24 5b b3 \scb{c_lightyellow}{04 1a 02 01}
\end{alltt}

The SP register will have started the same as before (at \verb@0x00000100@) 
and finished the same as before as well (at \verb@0x000000fc@.) 
\end{tcolorbox}



%%%%%%%%%%%%%%%%%%%%%%%%%%%%%%%%%%%%%%%%%%%%%%%%%%%%%%%%%%%%%%%%%%%%%%%%%%%%
%%%%%%%%%%%%%%%%%%%%%%%%%%%%%%%%%%%%%%%%%%%%%%%%%%%%%%%%%%%%%%%%%%%%%%%%%%%%
\subsubsection{Full Descending Stack: Pop}

If the SP register contains the value \verb@0x000000fc@ and the memory contains:

\begin{alltt}
000000e0  c9 b9 6e 41 b2 85 af 8d  c8 71 dd 31 12 2b 2c f9
000000f0  78 31 b8 c2 a7 d0 de 27  47 24 5b b3 \scb{c_lightyellow}{01 02 1a 04}
00000100  \scb{c_lightyellow}{91 de b5 9e} \scb{c_lightyellow}{e1 90 d4 b5}  06 d7 66 ad 79 dc 17 fb
\end{alltt}

\ldots\ then if we {\em pop} three 32-bit values out of the stack into registers A, B, and C:

\begin{tabular}{ll}
{\tt A $\leftarrow$ mem32(SP)}     & // load from memory address in SP register\\
{\tt SP $\leftarrow$ SP + 4}     & // {\em post}-increment using operand size SP=\verb@0x00000100@\\
{\tt B $\leftarrow$ mem32(SP)}     & // load from memory address in SP register\\
{\tt SP $\leftarrow$ SP + 4}     & // {\em post}-increment using operand size SP=\verb@0x00000104@\\
{\tt C $\leftarrow$ mem32(SP)}     & // load from memory address in SP register\\
{\tt SP $\leftarrow$ SP + 4}     & // {\em post}-increment using operand size SP=\verb@0x00000108@
\end{tabular}

\ldots\ then the SP register will contain \verb@0x00000108@, 
A=\verb@0x01021a04@, B=\verb@0x91deb59e@, C=\verb@0xe190d4b5@ and 
the contents of the memory will not have been changed by the pop operations.

\begin{tcolorbox}
Note that if this same example were run on a {\em little-endian} machine
starting with the same register and memory contents as a big-endiam example above 
then the three pop instructions would have resulted in SP=\verb@0x00000108@,
A=\verb@0x041a0201@, B=\verb@0x9eb5de91@, C=\verb@0xb5d490e1@ and, again,
the contents of the memory will not have been changed by the pop operations.
\end{tcolorbox}


%%%%%%%%%%%%%%%%%%%%%%%%%%%%%%%%%%%%%%%%%%%%%%%%%%%%%%%%%%%%%%%%%%%%%%%%%%%%
%%%%%%%%%%%%%%%%%%%%%%%%%%%%%%%%%%%%%%%%%%%%%%%%%%%%%%%%%%%%%%%%%%%%%%%%%%%%
\subsubsection{Empty Descending Stack: Push}

{\em Note that the difference between a Full Descending and an Empty Descending stack 
is simply the order in which the operations are performed in the RTL (pseudocode)
that describes when the SP register is decremented and when the values are stored
into the stack.}

If an {\em Empty Descending} stack is employed on a {\em big-endian} machine
with the following memory contents:
\begin{verbatim}
000000e0  c9 b9 6e 41 b2 85 af 8d  c8 71 dd 31 12 2b 2c f9
000000f0  78 31 b8 c2 a7 d0 de 27  47 24 5b b3 e0 e9 3b e2
00000100  91 de b5 9e e1 90 d4 b5  06 d7 66 ad 79 dc 17 fb
\end{verbatim}
and the SP (stack pointer) register is set to \verb@0x00000100@ then 
a {\em push} of the 32-bit value \verb@0x01021a04@ onto the stack would be 
performed like this:

\begin{tabular}{ll}
{\tt mem32(SP) $\leftarrow$ 0x01021a04}     & // store the 32-bit value into memory\\
{\tt SP $\leftarrow$ SP - 4}              & // {\em post}-decrement using operand size SP=\verb@0x000000fc@
\end{tabular}

Resulting in changing the contents of the four bytes at address 
\verb@0x00000100@ to this:
\begin{alltt}
000000e0  c9 b9 6e 41 b2 85 af 8d  c8 71 dd 31 12 2b 2c f9
000000f0  78 31 b8 c2 a7 d0 de 27  47 24 5b b3 e0 e9 3b e2
00000100  \scb{c_lightyellow}{01 02 1a 04} e1 90 d4 b5  06 d7 66 ad 79 dc 17 fb
\end{alltt}
\ldots\ and SP=\verb@0x000000fc@.


\begin{tcolorbox}
Note that if this same example were run on a {\em little-endian} machine, 
the only difference would be that the four bytes starting at memory address 
\verb@0x00000100@ would have been set to \verb@04 1a 02 01@ (rather than 
\verb@01 02 1a 04@) like this:

\begin{alltt}
00000100  \scb{c_lightyellow}{04 1a 02 01} e1 90 d4 b5  06 d7 66 ad 79 dc 17 fb
\end{alltt}

The SP register will have started the same as on a big-endian machine (at \verb@0x00000100@) 
and finished the same (at \verb@0x000000fc@.)
\end{tcolorbox}



%%%%%%%%%%%%%%%%%%%%%%%%%%%%%%%%%%%%%%%%%%%%%%%%%%%%%%%%%%%%%%%%%%%%%%%%%%%%
%%%%%%%%%%%%%%%%%%%%%%%%%%%%%%%%%%%%%%%%%%%%%%%%%%%%%%%%%%%%%%%%%%%%%%%%%%%%
\subsubsection{Empty Descending Stack: Pop}

{\em As seen in the Empty Descending Push example, the only difference between a 
Full Descending and an Empty Descending stack is the order in which the operations 
are performed in the RTL (pseudocode).}

If the SP register contains the value \verb@0x000000fc@ and the memory contains:

\begin{alltt}
000000e0  c9 b9 6e 41 b2 85 af 8d  c8 71 dd 31 12 2b 2c f9
000000f0  78 31 b8 c2 a7 d0 de 27  47 24 5b b3 01 02 1a 04
00000100  \scb{c_lightyellow}{91 de b5 9e} \scb{c_lightyellow}{e1 90 d4 b5}  \scb{c_lightyellow}{06 d7 66 ad} 79 dc 17 fb
\end{alltt}

\ldots\ then if we {\em pop} three 32-bit values out of the stack into registers A, B, and C:

\begin{tabular}{ll}
{\tt SP $\leftarrow$ SP + 4}     & // {\em pre}-increment using operand size SP=\verb@0x00000100@\\
{\tt A $\leftarrow$ mem32(SP)}     & // load from memory address in SP register\\
{\tt SP $\leftarrow$ SP + 4}     & // {\em pre}-increment using operand size SP=\verb@0x00000104@\\
{\tt B $\leftarrow$ mem32(SP)}     & // load from memory address in SP register\\
{\tt SP $\leftarrow$ SP + 4}     & // {\em pre}-increment using operand size SP=\verb@0x00000108@\\
{\tt C $\leftarrow$ mem32(SP)}     & // load from memory address in SP register
\end{tabular}

\ldots\ then the SP register will contain \verb@0x00000108@, 
A=\verb@0x91deb59e@, B=\verb@0xe190d4b5@, C=\verb@0x06d766ad@ and 
the contents of the memory will not have been changed by the pop operations.

\begin{tcolorbox}
Note that if this same example were run on a {\em little-endian} machine
starting with the same register and memory contents as a big-endiam example above 
then the three pop instructions would have resulted in SP=\verb@0x00000108@,
A=\verb@0x9eb5de91@, B=\verb@0xb5d490e1@, C=\verb@0xad66d706@ and, again,
the contents of the memory will not have been changed by the pop operations.
\end{tcolorbox}





%%%%%%%%%%%%%%%%%%%%%%%%%%%%%%%%%%%%%%%%%%%%%%%%%%%%%%%%%%%%%%%%%%%%%%%%%%%%
%%%%%%%%%%%%%%%%%%%%%%%%%%%%%%%%%%%%%%%%%%%%%%%%%%%%%%%%%%%%%%%%%%%%%%%%%%%%
%%%%%%%%%%%%%%%%%%%%%%%%%%%%%%%%%%%%%%%%%%%%%%%%%%%%%%%%%%%%%%%%%%%%%%%%%%%%
%%%%%%%%%%%%%%%%%%%%%%%%%%%%%%%%%%%%%%%%%%%%%%%%%%%%%%%%%%%%%%%%%%%%%%%%%%%%


\subsection{Ascending Stacks}

This section provides a detailed discussion of push and pop operations
on ascending stacks.

%%%%%%%%%%%%%%%%%%%%%%%%%%%%%%%%%%%%%%%%%%%%%%%%%%%%%%%%%%%%%%%%%%%%%%%%%%%%
%%%%%%%%%%%%%%%%%%%%%%%%%%%%%%%%%%%%%%%%%%%%%%%%%%%%%%%%%%%%%%%%%%%%%%%%%%%%
\subsubsection{Full Ascending Stack: Push}

{\em The only difference between a Full Ascending push and a Full Descending push 
is that the stack pointer is incremented rather than decremented in the RTL (pseudocode).}

If a {\em Full Ascending} stack is employed on a {\em big-endian} machine
with the following memory contents:
\begin{verbatim}
000000e0  c9 b9 6e 41 b2 85 af 8d  c8 71 dd 31 12 2b 2c f9
000000f0  78 31 b8 c2 a7 d0 de 27  47 24 5b b3 e0 e9 3b e2
00000100  91 de b5 9e e1 90 d4 b5  06 d7 66 ad 79 dc 17 fb
\end{verbatim}
and the SP (stack pointer) register is set to \verb@0x00000100@ then 
a {\em push} of the 32-bit value \verb@0x01021a04@ onto the stack would be 
performed like this:

\begin{tabular}{ll}
{\tt SP $\leftarrow$ SP + 4}              & //  pre-{\em increment} using operand size SP=\verb@0x00000104@\\
{\tt mem32(SP) $\leftarrow$ 0x01021a04}     & // store the 32-bit value into memory
\end{tabular}

Resulting in changing the contents of the four bytes at address 
\verb@0x00000104@ to this:
\begin{alltt}
000000e0  c9 b9 6e 41 b2 85 af 8d  c8 71 dd 31 12 2b 2c f9
000000f0  78 31 b8 c2 a7 d0 de 27  47 24 5b b3 e0 e9 3b e2
00000100  91 de b5 9e \scb{c_lightyellow}{01 02 1a 04}  06 d7 66 ad 79 dc 17 fb
\end{alltt}
\ldots\ and SP=\verb@0x00000104@.


\begin{tcolorbox}
Note that if this same example were run on a {\em little-endian} machine, 
the only difference would be that the four bytes starting at memory address 
\verb@0x00000104@ would have been set to \verb@04 1a 02 01@ (rather than 
\verb@01 02 1a 04@) like this:

\begin{alltt}
00000100  91 de b5 9e \scb{c_lightyellow}{04 1a 02 01}  06 d7 66 ad 79 dc 17 fb
\end{alltt}

The SP register will have started the same as on the big-endian machine (at \verb@0x00000100@) 
and finished the same as well (at \verb@0x00000104@.) 
\end{tcolorbox}


%%%%%%%%%%%%%%%%%%%%%%%%%%%%%%%%%%%%%%%%%%%%%%%%%%%%%%%%%%%%%%%%%%%%%%%%%%%%
%%%%%%%%%%%%%%%%%%%%%%%%%%%%%%%%%%%%%%%%%%%%%%%%%%%%%%%%%%%%%%%%%%%%%%%%%%%%
\subsubsection{Full Ascending Stack: Pop}

{\em The only difference between a Full Ascending pop and a Full Descending pop 
is that the stack pointer is decremented rather than incremented in the RTL (pseudocode).}

If the SP register contains the value \verb@0x00000100@ and the memory contains:

\begin{alltt}
000000e0  c9 b9 6e 41 b2 85 af 8d  c8 71 dd 31 12 2b 2c f9
000000f0  78 31 b8 c2 a7 d0 de 27  \scb{c_lightyellow}{47 24 5b b3} \scb{c_lightyellow}{01 02 1a 04}
00000100  \scb{c_lightyellow}{91 de b5 9e} e1 90 d4 b5  06 d7 66 ad 79 dc 17 fb
\end{alltt}

\ldots\ then if we {\em pop} three 32-bit values out of the stack into registers A, B, and C:

\begin{tabular}{ll}
{\tt A $\leftarrow$ mem32(SP)}     & // load from memory address in SP register\\
{\tt SP $\leftarrow$ SP - 4}     & // {\em post}-increment using operand size SP=\verb@0x000000fc@\\
{\tt B $\leftarrow$ mem32(SP)}     & // load from memory address in SP register\\
{\tt SP $\leftarrow$ SP - 4}     & // {\em post}-increment using operand size SP=\verb@0x000000xf8@\\
{\tt C $\leftarrow$ mem32(SP)}     & // load from memory address in SP register\\
{\tt SP $\leftarrow$ SP - 4}     & // {\em post}-increment using operand size SP=\verb@0x000000f4@
\end{tabular}

\ldots\ then the SP register will contain \verb@0x000000f4@, 
A=\verb@0x91deb59e@, B=\verb@0x01021a04@, C=\verb@0xe47245bb3@ and 
the contents of the memory will not have been changed by the pop operations.

\begin{tcolorbox}
Note that if this same example were run on a {\em little-endian} machine
starting with the same register and memory contents as a big-endiam example above 
then the three pop instructions would have resulted in SP=\verb@0x000000f4@,
A=\verb@0x9eb5de91@, B=\verb@0x041a0201@, C=\verb@0xb35b2447@ and, again,
the contents of the memory will not have been changed by the pop operations.
\end{tcolorbox}



%%%%%%%%%%%%%%%%%%%%%%%%%%%%%%%%%%%%%%%%%%%%%%%%%%%%%%%%%%%%%%%%%%%%%%%%%%%%
%%%%%%%%%%%%%%%%%%%%%%%%%%%%%%%%%%%%%%%%%%%%%%%%%%%%%%%%%%%%%%%%%%%%%%%%%%%%
\subsubsection{Empty Ascending Stack: Push}

{\em The only difference between a Full Ascending push and an Empty Ascending push 
is the order in which the operations are performed in the RTL (pseudocode)
that describe when the SP register is incremented and when the values are stored
into the stack.}

If an {\em Empty Ascending} stack is employed on a {\em big-endian} machine
with the following memory contents:
\begin{verbatim}
000000e0  c9 b9 6e 41 b2 85 af 8d  c8 71 dd 31 12 2b 2c f9
000000f0  78 31 b8 c2 a7 d0 de 27  47 24 5b b3 e0 e9 3b e2
00000100  91 de b5 9e e1 90 d4 b5  06 d7 66 ad 79 dc 17 fb
\end{verbatim}
and the SP (stack pointer) register is set to \verb@0x00000100@ then 
a {\em push} of the 32-bit value \verb@0x01021a04@ onto the stack would be 
performed like this:

\begin{tabular}{ll}
{\tt mem32(SP) $\leftarrow$ 0x01021a04}     & // store the 32-bit value into memory\\
{\tt SP $\leftarrow$ SP + 4}              & //  post-{\em increment} using operand size SP=\verb@0x00000104@
\end{tabular}

Resulting in changing the contents of the four bytes at address 
\verb@0x00000104@ to this:
\begin{alltt}
000000e0  c9 b9 6e 41 b2 85 af 8d  c8 71 dd 31 12 2b 2c f9
000000f0  78 31 b8 c2 a7 d0 de 27  47 24 5b b3 e0 e9 3b e2
00000100  \scb{c_lightyellow}{01 02 1a 04} e1 90 d4 b5  06 d7 66 ad 79 dc 17 fb
\end{alltt}
\ldots\ and SP=\verb@0x00000104@.


\begin{tcolorbox}
Note that if this same example were run on a {\em little-endian} machine, 
the only difference would be that the four bytes starting at memory address 
\verb@0x00000100@ would have been set to \verb@04 1a 02 01@ (rather than 
\verb@01 02 1a 04@) like this:

\begin{alltt}
00000100  \scb{c_lightyellow}{04 1a 02 01} e1 90 d4 b5  06 d7 66 ad 79 dc 17 fb
\end{alltt}

The SP register will have started the same as on the big-endian machine (at \verb@0x00000100@) 
and finished the same as well (at \verb@0x00000104@.) 
\end{tcolorbox}


%%%%%%%%%%%%%%%%%%%%%%%%%%%%%%%%%%%%%%%%%%%%%%%%%%%%%%%%%%%%%%%%%%%%%%%%%%%%
%%%%%%%%%%%%%%%%%%%%%%%%%%%%%%%%%%%%%%%%%%%%%%%%%%%%%%%%%%%%%%%%%%%%%%%%%%%%
\subsubsection{Empty Ascending Stack: Pop}

{\em The only difference between an Empty Ascending pop and a Full Ascending pop 
is the order in which the operations are performed in the RTL (pseudocode)
that describe when the SP register is decremented and when the values are stored
into the stack.}

If the SP register contains the value \verb@0x000000fc@ and the memory contains:

\begin{alltt}
000000e0  c9 b9 6e 41 b2 85 af 8d  c8 71 dd 31 12 2b 2c f9
000000f0  \scb{c_lightyellow}{78 31 b8 c2} \scb{c_lightyellow}{a7 d0 de 27}  \scb{c_lightyellow}{47 24 5b b3} 01 02 1a 04
00000100  91 de b5 9e e1 90 d4 b5  06 d7 66 ad 79 dc 17 fb
\end{alltt}

\ldots\ then if we {\em pop} three 32-bit values out of the stack into registers A, B, and C:

\begin{tabular}{ll}
{\tt SP $\leftarrow$ SP - 4}     & // {\em pre}-decrement using operand size SP=\verb@0x000000f8@\\
{\tt A $\leftarrow$ mem32(SP)}     & // load from memory address in SP register\\
{\tt SP $\leftarrow$ SP - 4}     & // {\em pre}-decrement using operand size SP=\verb@0x000000f4@\\
{\tt B $\leftarrow$ mem32(SP)}     & // load from memory address in SP register\\
{\tt SP $\leftarrow$ SP - 4}     & // {\em pre}-decrement using operand size SP=\verb@0x000000f0@\\
{\tt C $\leftarrow$ mem32(SP)}     & // load from memory address in SP register
\end{tabular}

\ldots\ then the SP register will contain \verb@0x000000f0@, 
A=\verb@0x47245bb3@, B=\verb@0xa7d0de27@, C=\verb@0x7831b8c2@ and 
the contents of the memory will not have been changed by the pop operations.

\begin{tcolorbox}
Note that if this same example were run on a {\em little-endian} machine
starting with the same register and memory contents as a big-endiam example above 
then the three pop instructions would have resulted in SP=\verb@0x000000f0@,
A=\verb@0xb35b2447@, B=\verb@0x27ded0a7@, C=\verb@0xc2b83178@ and, again,
the contents of the memory will not have been changed by the pop operations.
\end{tcolorbox}





\end{document}
